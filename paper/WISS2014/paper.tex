%
%
%
\documentclass[twoside]{wiss}

\usepackage{graphicx}
\usepackage{nidanfloat} %% appended in WISS2010 for Future Vision (2010/7/7:akita)
\usepackage{multicol}
\usepackage{here} % [H]とするとその場所に配置されるらしい
\usepackage{color}

%% balance.styを追加 (2012/9/27:watanabe, Igarashi)
\usepackage{balance}    %% 最後のページの高さを揃えるために追加  (2012/9/27:watanabe, Igarashi)
%%% 最後のページの2段組の高さを揃える.\balanceを入れる.
%%% そろえたくないときは,\nobalance

\def\GEAR{\textsf{Gear}}
\def\GB{\textsf{GearBrowser}}
\def\figwidth{50mm}

% \def\▲{\textcolor{green}{▲}}
% \def\▼{\textcolor{green}{▼}}
\def\uptriangle{ \includegraphics[width=3mm,bb=0 0 36 36]{figures/uptriangle.pdf} }
\def\downtriangle{ \includegraphics[width=3mm,bb=0 0 36 36]{figures/downtriangle.pdf} }
\def\righttriangle{ \includegraphics[width=3mm,bb=0 0 36 36]{figures/righttriangle.pdf} }
\def\lefttriangle{ \includegraphics[width=3mm,bb=0 0 36 36]{figures/lefttriangle.pdf} }
\long\def\comment#1{}

\journalhead{「超」ナビゲーション}

\begin{document}

\title{「超」ナビゲーション}
\etitle{} %2012年では英文タイトルは廃止されました.記入しないでください
% Supernavigation

% \author{増井 俊之\affil{Toshiyuki Masui, 慶應義塾大学 環境情報学部}}
\author{■■ ■■\affil{■■■ ■■■}}

\begin{abstract}
単純な装置で大規模な階層データを効率的にナビゲーションする手法を提案する。
%
% ファイルシステムや住所のような
大規模な階層データから項目を選ぶ場合、
% 階層を上下に移動したり同じ階層内の選択項目を移動したりするのが普通である。
階層を移動したり階層内の選択項目を移動したりすることによって目的の項目を検索するのが一般的である。
たとえば日本の住所リストから「■■県●●市▲▲」を検索する場合、
都道府県リストから■■県を選択し/■■県の市町村リストから●●市を選択し/
といった操作を繰り返して目的の項目に到達する。
別の県や市町村の住所を選択する場合は上位層に移動してから同様の操作を行なう。
%
このようなナビゲーションを行なうためには、
階層を上下に移動する手段と階層内を移動する手段が必要になるため、
3個以上のキーが用いられるのが普通である。

本論文では、2個のキーだけを使って
階層データのナビゲーションを実現する手法「{\GEAR}」を提案する。
{\GEAR}では
(1)階層内の項目選択時に端まで来た場合は上の階層に移動する.
(2)選択中の項目に下位階層がある場合は一定時間後に下の階層に移動する.
という手法により、
階層を上下に移動するためのキーが不要になり、
2個のキーだけであらゆる階層データを効率的にナビゲーションすることが可能になる。
\end{abstract}

\maketitle

\section{はじめに}

% 大量のデータを扱うために階層的に整理する手法がよく利用されている。

ファイルシステム、
Web、
住所のような
% バーコードで表現される商品番号など、
大規模データの多くは階層構造として表現されており、
%
% Unixで導入された階層型ディレクトリは現在あらゆるパソコンで導入されている。
%
% 階層型ファイルシステムはUnixで導入されたものであるが、
% その利便性のために現在は他のOSでも標準的になっており、
% 階層を「/」で表現するUnixのファイル記法がWebのURL記法として標準になっている。
%
% URL、ファイル名、住所のような大規模データは木構造で表現されることが多い。
%
% 住所データのように本質的に階層的なものもあるし、
% URLやパソコンのファイル名は階層的な名前を持っているし、
% 世の中の多くの情報は階層的な構造を持っており、
%
% 写真も日付により階層構造である
%
% (図を描く.....)
%
階層構造を利用して情報を検索するインタラクション手法が広く利用されている。
階層的なデータを扱うための様々な情報視覚化手法が提案されているが
\cite{Johnson:1991:TSA:949607.949654}\cite{Lamping:1995:FTB:223904.223956}\cite{Stasko:2000:FDN:857190.857683}、
これらはまだ一般的には普及しておらず、
%  Treemapとか
現在のパソコンや携帯機器では、
シンプルなマウス操作やキー操作で階層構造をたどる手法が広く使われている。
%
例えばMacのファインダ\footnote{
  Macのデスクトップ画面でファイルを操作するための常駐基本ソフトウェア
}では、
ファイルの階層構造を視覚化/ナビゲーションするために
複数の手法が用意されており、
シンプルなマウス/キー操作で
階層型ファイルシステムのナビゲーションを行なうことができるようになっている。

%...みたいなキー操作で階層情報をたどることができる。

% 木構造のノードをたどることによって
% そこから必要な情報を取得するための様々な手法が利用されている。

パソコンや携帯機器のキーやボタンを利用して階層構造データのナビゲーションを行なう場合、
階層を上下に移動したり、
% 兄弟関係にある
項目のリスト内を移動したりすることによって目的の情報を捜すのが普通である。
たとえばMacのファインダでは
上下矢印キーを使ってファイルやフォルダを選択したり、
左右矢印キーを使って階層を移動したりすることによって
目的のファイルに到達できるようになっている。

% (提案)

このような手法でナビゲーションを行なうためには、
通常3個以上のキーやスイッチが必要になる。
%
ファインダやテレビのリモコンなどでは
上下左右4方向のキーで階層データのナビゲーションを行なうようになっているし、
%以前のソニーの携帯電話に登載されていたジョグダイヤルでは、
ジョグダイヤルを登載した携帯電話や携帯端末では、
ダイヤルを回す操作とダイヤルを押す操作を組み合わせて
項目を選択したり階層を移動したりできるものがある。

一方、2個のスイッチだけで階層情報のナビゲーションを実行することができれば、
4方向キーや押しボタンつきジョグダイヤルなどよりも
はるかに単純な装置を使って階層情報のナビゲーションが可能になり、
いつでもどこでも誰でも簡単にデータを検索することができるようになる可能性がある。
%
本論文では、2個のスイッチだけを使って
階層構造を効率的にナビゲーションする「{\GEAR}」システムについて述べる。

% これらの装置を総称して「{\GEAR}」と呼ぶことにする。

% {\GEAR}では、{\uptriangle}と{\downtriangle}というふたつのキーのみを利用して階層データのナビゲーションを行なう。

\section{{\GEAR}のナビゲーション}
\label{description}

% (1) 階層内の項目選択時に端まで来た場合は上の階層に移動する
% (2) 選択中の項目に下位階層がある場合は一定時間後に下の階層に移動する
% 
% Macのファインダを利用して{\GEAR}の動きを説明する。

以下のような階層をもつファイルシステムのナビゲーションを考える。

\begin{figure}[H]
\centerline{\includegraphics[width=65mm,bb=0 0 509 502]{figures/ae9216b00626f9c4eea44cc380f25886.png}}
\caption{階層的に表現されたショッピングモールの店リスト}
\label{screenshot1}
\end{figure}

\subsection{Macのファインダ上での階層情報ナビゲーション}

Macのファインダでは
{\uptriangle}{\downtriangle}{\lefttriangle}{\righttriangle}という4個の矢印キーで
ファイルシステムのナビゲーションを行なうことができる。

「店リスト」をファインダで表示して「本屋」を選択すると、表示は以下のようになる。

\begin{figure}[H]
\centerline{\includegraphics[width=\figwidth,bb=0 0 344 272]{figures/9b121bec45e5b480e5ac64fdd0f82592.png}}
\caption{店リストから「本屋」を選択}
\label{screenshot2}
\end{figure}

\noindent
ここで{\downtriangle}を押すと、次の「文房具屋」が選択される。

\begin{figure}[H]
\centerline{\includegraphics[width=\figwidth,bb=0 0 344 272]{figures/f43016d1b524baf414f2c32c48fe9588.png}}
\caption{ssss}
\label{「文房具屋」を選択}
\end{figure}

\noindent
さらに二回{\downtriangle}を押すと、以下のように「食料品店」が選択される。

\begin{figure}[H]
\centerline{\includegraphics[width=\figwidth,bb=0 0 344 272]{figures/c074cd6daec3da0341125d1492b8a09c.png}}
\caption{「食料品店」を選択}
\label{screenshot4}
\end{figure}

\noindent
「食料品店」は下位階層を持っているので、
{\righttriangle}キーを押すことによって以下のように下位階層が表示される。

\begin{figure}[H]
\centerline{\includegraphics[width=\figwidth,bb=0 0 344 272]{figures/51d867d4721f65c18e84172c8818e137.png}}
\caption{「食料品店」の下位階層を展開して表示}
\label{screenshot5}
\end{figure}

\noindent
ここで{\downtriangle}キーを押すことによって「酒屋」を選択したり、
「生鮮食料品店」を選択してから{\righttriangle}を押すことによって、
さらに下位階層を表示することができる。

\begin{figure}[H]
\centerline{\includegraphics[width=\figwidth,bb=0 0 344 298]{figures/ce3ee682612de44d6c663a7323c262a6.png}}
\caption{「生鮮食料品店」の下位階層を表示}
\label{screenshot6}
\end{figure}

\noindent
また、この状態で{\lefttriangle}キーを押すと下位層の表示を消し、
図\ref{screenshot5}の状態に戻すことができる。

このように、Macのファインダでは4個のキーを使って
階層データのナビゲーションを行なうことができる。
テレビのリモコンやジョグダイヤルでもほぼ同様の手法が利用されている。

\subsection{{\GEAR}によるナビゲーション}

{\GEAR}では{\uptriangle}と{\downtriangle}というふたつのキーだけを
利用してナビゲーションを行なう。

{\GEAR}で「店リスト」を表示すると、
ファインダの場合と同じように
図\ref{screenshot2}のように表示される。
{\downtriangle}を3回押すと
図\ref{screenshot42}のように「食料品店」が選択されるが、
そのまま一定時間待つと「食料品店」の下位層が自動的に展開されて、
図\ref{screenshot7}のようにその最初の要素が選択される。

\begin{figure}[H]
\centerline{\includegraphics[width=\figwidth,bb=0 0 344 272]{figures/c074cd6daec3da0341125d1492b8a09c.png}}
\caption{「食料品店」を選択}
\label{screenshot42}
\end{figure}

\begin{figure}[H]
\centerline{\includegraphics[width=\figwidth,bb=0 0 344 272]{figures/2387e402f81dbe7917e04df82b0a659c.png}}
\caption{「食料品店」の下位階層を自動展開}
\label{screenshot7}
\end{figure}

\noindent
ここで{\downtriangle}を2回押して「生鮮食料品店」を選択したまま一定時間待つと、
図\ref{screenshot8}のように
下位層が自動的に展開され、最初の要素である「魚屋」が選択される。

\begin{figure}[H]
\centerline{\includegraphics[width=\figwidth,bb=0 0 344 304]{figures/1b1955309d3baefda8e1b614cf06df62.png}}
\caption{「生鮮食料品店」の下位階層を自動展開}
\label{screenshot8}
\end{figure}

\noindent
つまり、{\righttriangle}のようなキーを押さなくても、
一定時間待つことによって同様の効果が得られることになる。

図\ref{screenshot42}のように食料品店を選択した状態から
時間を置かずに{\downtriangle}を押すと、
下位層は展開されず、次の「衣料品店」が選択される。

\begin{figure}[H]
\centerline{\includegraphics[width=\figwidth,bb=0 0 344 304]{figures/c5c757d8f79d5a8a9c85eef25600ba66.png}}
\caption{「衣料品店」を選択}
\label{screenshot9}
\end{figure}

\noindent
ここで操作を止めて一定時間待つと
下位層が自動的に展開されて「靴屋」が選択される。

\begin{figure}[H]
\centerline{\includegraphics[width=\figwidth,bb=0 0 344 304]{figures/fddd5777d39924ea3f0220ae39a604c1.png}}
\caption{「靴屋」を選択}
\label{screenshot10}
\end{figure}

図\ref{screenshot8}の状態から{\uptriangle}キーを押すと、
下位層は自動的に閉じられて図\ref{screenshot7}の状態に戻る。 
さらに{\uptriangle}キーを押すと
「食料品店」の下位の層も閉じられ、図\ref{screenshot42}の状態に戻る。
%
また、図\ref{screenshot8}の状態から{\downtriangle}を2回押すと
「食料品店」の下位層は自動的に閉じられて図\ref{screenshot9}の状態になる。

まとめると、

\begin{enumerate}
\item \textbf{選択した項目に下位層が存在するときキー入力を行なわずに待つと下位層が自動的に展開され、下位層の最初の項目が選択される}
\item \textbf{項目リストの端を選択しているとき、さらに{\uptriangle}{\downtriangle}キーを押すと下位層は閉じられてひとつ上の層の項目が選択される}
\end{enumerate}

\noindent
というふたつの工夫により、
{\uptriangle}と{\downtriangle}だけで
階層データを自由にナビゲーションすることが可能になる。

\section{実装}

ブラウザにJavaScriptで{\GEAR}を実装した
「{\GB}」を図\ref{gearbrowser}に示す。
ニュース、動画、音楽、電子書籍、レシピ、地図など
ブラウザで表示可能な多数のコンテンツの目次を{\GEAR}ウィンドウとして左側に表示し、
右側にコンテンツを表示している。

\begin{figure*}
\centerline{\includegraphics[width=160mm,bb=0 0 1401 872]{figures/ab4ff7c2d44f4af2bb94fae76589f495.png}}
\caption{\textsf{GearBrowser}}
\label{gearbrowser}
\end{figure*}

ユーザの操作は{\uptriangle}と{\downtriangle}のみである。
上下矢印キーの他、
マウスホイールの回転も{\uptriangle}と{\downtriangle}として利用できるので、
ワイヤレスマウスをリモコンのように利用することができる。

著者は自宅の居間の大型テレビに接続したMac miniで{\GB}を半年以上利用している。
筆者宅にはデジタル地上波が届かないこともあり、
{\GB}だけを利用して各種のコンテンツを楽しんでいる。

\section{議論}

\subsection{適用可能なデータのサイズ}

\ref{description}章では小さな階層データを利用して{\GEAR}の動作の説明を行なったが、
{\GEAR}で巨大なデータを扱うことに問題はない。
ファインダで扱えるように適切な階層構造が構築されていれば
{\GEAR}でも同様にナビゲーションが可能である。

{\GB}ではすべての青空文庫コンテンツや1万本以上のアニメ動画を
問題なくナビゲーションしている。

%   普通にあらゆるデータに使えるといえるだろう
%   10レベルで6階層あれば10\^6のコンテンツに対応できるとか

\subsection{音声の利用}

{\GEAR}は階層構造表示しながら利用するのが基本であるが、
項目を選択したときタイトルを読みあげることにより、
階層構造を表示せずにナビゲーションを行なうことが可能である。
この場合は階層の構造についてあらかじめ知っておくことが望ましいが、
表示装置を利用できない状況でも
音楽コンテンツを選択可能になるので便利である。

\subsection{入力装置いろいろ}

PowerMate、円板、ローラー、パドル

\subsection{操作の数}

図\ref{screenshot2}の状態から
図\ref{screenshot8}の状態に移動する場合、
ファインダでは9回キー操作を行なう必要があるが
{\GEAR}では5回だけでよい。
{\GEAR}では下位層に移動するのに時間待ちの必要があるので操作全体にかかる時間は大差ないが、
操作の数が少なくてすむのは確かなので、
運動障害のあるユーザや
操作をしにくい環境において有効と考えられる。

\subsection{扱えるデータ構造}

辞書のようなデータは読みや綴りで階層的に分類できるし、
時刻情報のような連続的なものでも
年/月/日/時間/分/秒のような階層で表現することができるので

辞書のようなフラットなデータや時刻のような連続的なデータも
読みなどを利用して階層的に表現可能であるし、
SNSの友達関係のようなネットワーク構造も
木構造的に表現することは可能なので、
ほぼあらゆるデータは木構造で表現可能だといえる。

\subsection{想定される利用者}

著者のひとりは{\GEAR}を通常の家庭環境で利用しているが、

  はっきり言って、ほとんど誰でも使えると思う

  Unixでよく使うコマンドは cd, ls, more とかである
  FinderとかExplorerとかの基本ソフトで使う
  これ以外にはキーワード検索とかもあるけど

\subsection{ザッピング}

% 決定ボタンを押す方式だと、単に回転するだけでは次のページに移ることができない

\begin{figure}[H]
\centerline{\includegraphics[width=\figwidth,bb=0 0 344 318]{figures/9a8615b0242c9ba4deb77ca30ab94d7c.png}}
\caption{1巻の最終ページ}
\label{manga1}
\end{figure}

{\GB}では、コンテンツを単純な操作で連続的に楽しむことができる。
%
図\ref{manga1}のような漫画の1巻の最終ページから
次の巻の最初のページに移りたいとき、
ファインダのように上下左右キーを利用する場合は

\begin{itemize}
\item {\lefttriangle}で1巻を閉じる
\item {\downtriangle}で2巻を選択する
\item {\righttriangle}で2巻の要素を開く
\item {\downtriangle}で2巻の先頭要素を選択する
\end{itemize}

\noindent
という操作が必要であるが、{\GB}では

\begin{itemize}
\item {\downtriangle}を押す
\end{itemize}

\noindent
だけでよい。
図\ref{manga1}の状態で{\downtriangle}を押すと
1巻の下位要素は閉じられ、
2巻の下位要素が自動的に開いて最初の要素が表示されるからである。

つまり、{\GB}では
何も考えずに
{\downtriangle}を押すだけでコンテンツを順番に楽しむことができることになる。
何も考えずに単純な操作を繰り返すだけでコンテンツを検索できるということは
従来のテレビのチャンネルを回す「ザッピング」と似ており、
受動的な人でも楽しめるということだろう。

\subsection{操作が超シンプルだというのはどういうことか}

  かなりの運動障害があっても使える
  操作が少ないので、誰でも試行錯誤でなんとかなるだろう
  一方、筆者は毎日食卓から{\GEAR}を使ってニュースを見たり音楽を聞いたりしている
  障害があってもなくても同じように便利だというのが理想である

\subsection{テレビとWebの融合が失敗した理由}
   そもそも姿勢が違うのである
   前かがみになって入力装置を操作するのはテレビには向いいない

\subsection{ひとつの階層に沢山のデータを置いてはいけない。スクロールの速度にもよるが、20個以下程度が妥当と思われる。}
   読みの場合は「あ」「か」「さ」

\subsection{どういうデータに使えるか}
   {\GEAR}は木構造にしか適用できない
   専門的に管理されているデータは木構造が多く、
   そうでないものはタグなどを利用してネットワーク構造にするのが良いと思われる
   あらゆるネットワーク構造は木構造に変換可能だし
     表型式でも木構造の一種と考えることができる
     Cでは「配列の配列」が「表」だが
   同じデータがあちこちに出てきてもよい
     日付による分類とキーワードによる分類

\subsection{ザッピングについて}
   下矢印だけで次頁に移動できる!
   最下位の層でザッピングできる
   単に「次」を押していけば漫画や順番に
   「前」の場合はそうはならないが、「次」の方が圧倒的に多いはずだから大丈夫

\subsection{どういうデバイスを使うか}

  「次」「前」だけでいいから2個のスイッチだけでよい
   圧力センサやローラーなど、各週の実装が可能である
   (写真いろいろ)

\subsection{障害対応}

「障害者」用の装置を作っては駄目で、障害が有ってもなくても使える装置を作るべきだろう。

同様に、初心者用/老人用/子供用のシステムを作るよりも、初心者でも老人で
も子供でも使えるシステムを作る方がスジが良いと思う。

\subsection{時間を使うことはどうなのか}

ALS用文字入力システムではタイミングで選択操作を行なっている
  何と呼ぶのだっけ?
  腕は自由に動かないが、あるタイミングで腕を動かすことは可能だというタイプの障害が存在し、
    そういう症状は多い
e.g. Pete

\subsection{選択以外の操作}

普通のWebページだとスクロール操作が必要になるが
  スクロールはなんとか大丈夫
  早送りとかは工夫が必要
    だけどできる
音量や画面の明るさ調整などは別操作が必要になるだろう
     これは別のチャンネルを利用すべきかもしれない
     別になっている方が一般に望ましいだろう
       だからiPhoneでもAndroidでも音量だけは別操作になっていてメニューで選んだりしない

\subsection{速度について}
   もちろんパソコン上の操作より遅い
   装置が簡単だということがメリットだろう
     トレードオフである
   マウスが使える状況では無理に{\GEAR}を使う必要などないだろう
     ズーミングでもなんでも使えばいい

\subsection{Lindaで実装}

\section{結論}

非常に単純な入力装置を利用して大規模な階層構造データをナビゲーションする
手法「{\GEAR}」を提案した。

{\GEAR}の使い方は単純であり、
一度慣れてしまえば問題なく使えることは確認できている。

{\GEAR}はあまりにも単純な手法であるため、
同じ手法がこれまでに存在した可能性を否定することはできないが、
半年以上の間、多くのHI研究者や開発者に意見/感想を求めたところ
{\GEAR}と同じ手法の存在は確認できていないので、
少なくとも近年同様のシステムが存在した可能性は低いと思われる。

{\GEAR}のように実装も操作法も簡単で有用なシステムは、
誰もがいつでもどこでも計算機やネットワークを活用する
ユビキタスコンピューティング社会において
重要な存在になるであろう。

{\scriptsize
\bibliographystyle{jwiss}
\bibliography{paper}
}

\end{document}



